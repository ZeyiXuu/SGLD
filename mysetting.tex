\usepackage{amsmath, amssymb, amsthm,graphicx}
\usepackage[font=small,labelfont=md,textfont=it]{caption}
\usepackage{floatrow,float}
\usepackage[titletoc, title]{appendix}
\usepackage[colorlinks,linkcolor=blue,citecolor=blue]{hyperref}
\usepackage[hyperpageref]{backref}
\usepackage{longtable}
\usepackage{pgfplots}
\usepackage{diagbox}
\usepackage{booktabs,makecell,multirow}
\usepackage[capitalise, nosort]{cleveref}
\usepackage{cases,color}
\usepackage{algorithm}
\usepackage{algorithmic}
\usepackage{amsmath,bm}
%\usepackage{geometry}

\usepackage{tcolorbox}
\usepackage{array}
\renewcommand\arraystretch{1.5}

\newtheorem{algo}{Algorithm}
%\geometry{left = 3cm,right=3cm,top =3cm,bottom = 3.5cm }

\DeclareMathOperator{\D}{D}
\DeclareMathOperator{\I}{I}

\crefname{equation}{}{}
\crefname{lem}{Lemma}{Lemmas}
\crefname{thm}{Theorem}{Theorems}

\newcommand{\dd}{\,{\rm d}}
\newcommand{\R}{\,{\mathbb R}}
\newcommand{\bs}{\boldsymbol}
\newcommand{\dual}[1]{\left\langle {#1} \right\rangle}
\newcommand{\proxi}[0]{ {\bf prox}}
\newcommand{\J}[0]{ {\bf J}}
\newcommand{\dom}[0]{ {\bf dom\,}}
\newcommand{\argmin}[0]{ {\mathrm {argmin}\,}}
\newcommand{\inner}[1]{\left({#1} \right)}
\newcommand{\jmp}[1]{{[\![ {#1} ]\!]}}
\newcommand{\nm}[1]{\left\lVert {#1} \right\rVert}
\newcommand{\snm}[1]{\left\lvert {#1} \right\rvert}
\newcommand{\ssnm}[1]
{
  \left\vert\kern-0.25ex
  \left\vert\kern-0.25ex
  \left\vert
  {#1}
  \right\vert\kern-0.25ex
  \right\vert\kern-0.25ex
  \right\vert
}
\newcommand{\lsnm}[1]{\snm{#1}^{\text{\tiny L}}}
\newcommand{\rsnm}[1]{\snm{#1}^{\text{\tiny R}}}

\renewcommand{\algorithmicrequire}{ \textbf{Input:}} %Use Input in the format of Algorithm
\renewcommand{\algorithmicensure}{ \textbf{Output:}} %UseOutput in the format of Algorithm

\makeatletter
\def\spher@harm#1{%
  \vbox{\hbox{%
    \offinterlineskip
    \valign{&\hb@xt@2\p@{\hss$##$\hss}\vskip.2ex\cr#1\crcr}%
  }\vskip-.36ex}%
}
\def\gshone{\spher@harm{.}}
\def\gshtwo{\spher@harm{.&.}}
\def\gshthree{\spher@harm{.&.&.}}
\let\gsh\spher@harm
\makeatother

\newtheorem{theorem}{Theorem}[section]
\newtheorem{lemma}[theorem]{Lemma}
\newtheorem{corollary}[theorem]{Corollary}
\newtheorem{proposition}[theorem]{Proposition}
\newtheorem{definition}[theorem]{Definition}
\newtheorem{example}[theorem]{Example}
\newtheorem{exercise}[theorem]{Exercise}
\newtheorem{question}[theorem]{Question}
\newtheorem{remark}[theorem]{Remark}
\newtheorem{alg}[theorem]{Algorithm}

%\newtheorem{algor}{Algorithm}
% \newtheorem{alg}{algorithm}
% %\newtheorem{alg}[theorem]{Algorithm}
% \newtheorem{coro}{corollary}[section]
% \newtheorem{prop}{proposition}[section]
% \newtheorem{Def}{definition}[section]
% \newtheorem{assum}{assumption}
% \newtheorem{lem}{lemma}[section]
% \newtheorem{rem}{remark}[section]
% \newtheorem{thm}{theorem}[section]
% \newtheorem{example}{example}[section]
% \newtheorem{exercise}{exercise}

\newcommand{\LH}[1]{\textcolor{blue}{#1}}
\newcommand{\LC}[1]{\textcolor{red}{#1}}
\newcommand{\JW}[1]{\textcolor{cyan}{#1}}
  \newcounter{mnote}
\setcounter{mnote}{0}
\newcommand{\mnote}[1]{\addtocounter{mnote}{1}
	\ensuremath{{}^{\bullet\arabic{mnote}}}
	\marginpar{\footnotesize\em\color{red}\ensuremath{\bullet\arabic{mnote}}#1}}
\let\oldmarginpar\marginpar
\renewcommand\marginpar[1]{\-\oldmarginpar[\raggedleft\footnotesize #1]%
	{\raggedright\footnotesize #1}}

%\renewcommand\qedsymbol{\hfill\ensuremath{\blacksquare}}

\makeatletter\def\@captype{table}\makeatother