\documentclass[10pt]{amsart}
\usepackage{amsmath, amssymb, amsthm,graphicx}
\usepackage[font=small,labelfont=md,textfont=it]{caption}
\usepackage{floatrow,float}
\usepackage[titletoc, title]{appendix}
\usepackage[colorlinks,linkcolor=blue,citecolor=blue]{hyperref}
\usepackage[hyperpageref]{backref}
\usepackage{longtable}
\usepackage{pgfplots}
\pgfplotsset{compat=1.18}
\usepackage{diagbox}
\usepackage{booktabs,makecell,multirow}
\usepackage[capitalise, nosort]{cleveref}
\usepackage{cases,color}
\usepackage{algorithm}
\usepackage{algorithmic}
\usepackage{amsmath,bm}
%\usepackage{geometry}

\usepackage{tcolorbox}
\usepackage{array}
\renewcommand\arraystretch{1.5}

\newtheorem{algo}{Algorithm}
%\geometry{left = 3cm,right=3cm,top =3cm,bottom = 3.5cm }

\DeclareMathOperator{\D}{D}
\DeclareMathOperator{\I}{I}

\crefname{equation}{}{}
\crefname{lem}{Lemma}{Lemmas}
\crefname{thm}{Theorem}{Theorems}

\newcommand{\dd}{\,{\rm d}}
\newcommand{\R}{\,{\mathbb R}}
\newcommand{\bs}{\boldsymbol}
\newcommand{\dual}[1]{\left\langle {#1} \right\rangle}
\newcommand{\proxi}[0]{ {\bf prox}}
\newcommand{\J}[0]{ {\bf J}}
\newcommand{\dom}[0]{ {\bf dom\,}}
\newcommand{\argmin}[0]{ {\mathrm {argmin}\,}}
\newcommand{\inner}[1]{\left({#1} \right)}
\newcommand{\jmp}[1]{{[\![ {#1} ]\!]}}
\newcommand{\nm}[1]{\left\lVert {#1} \right\rVert}
\newcommand{\snm}[1]{\left\lvert {#1} \right\rvert}
\newcommand{\ssnm}[1]
{
  \left\vert\kern-0.25ex
  \left\vert\kern-0.25ex
  \left\vert
  {#1}
  \right\vert\kern-0.25ex
  \right\vert\kern-0.25ex
  \right\vert
}
\newcommand{\lsnm}[1]{\snm{#1}^{\text{\tiny L}}}
\newcommand{\rsnm}[1]{\snm{#1}^{\text{\tiny R}}}

\renewcommand{\algorithmicrequire}{ \textbf{Input:}} %Use Input in the format of Algorithm
\renewcommand{\algorithmicensure}{ \textbf{Output:}} %UseOutput in the format of Algorithm

\makeatletter
\def\spher@harm#1{%
  \vbox{\hbox{%
    \offinterlineskip
    \valign{&\hb@xt@2\p@{\hss$##$\hss}\vskip.2ex\cr#1\crcr}%
  }\vskip-.36ex}%
}
\def\gshone{\spher@harm{.}}
\def\gshtwo{\spher@harm{.&.}}
\def\gshthree{\spher@harm{.&.&.}}
\let\gsh\spher@harm
\makeatother

\newtheorem{theorem}{Theorem}[section]
\newtheorem{lemma}[theorem]{Lemma}
\newtheorem{corollary}[theorem]{Corollary}
\newtheorem{proposition}[theorem]{Proposition}
\newtheorem{definition}[theorem]{Definition}
\newtheorem{example}[theorem]{Example}
\newtheorem{exercise}[theorem]{Exercise}
\newtheorem{question}[theorem]{Question}
\newtheorem{remark}[theorem]{Remark}
\newtheorem{alg}[theorem]{Algorithm}

%\newtheorem{algor}{Algorithm}
% \newtheorem{alg}{algorithm}
% %\newtheorem{alg}[theorem]{Algorithm}
% \newtheorem{coro}{corollary}[section]
% \newtheorem{prop}{proposition}[section]
% \newtheorem{Def}{definition}[section]
% \newtheorem{assum}{assumption}
% \newtheorem{lem}{lemma}[section]
% \newtheorem{rem}{remark}[section]
% \newtheorem{thm}{theorem}[section]
% \newtheorem{example}{example}[section]
% \newtheorem{exercise}{exercise}

\newcommand{\LH}[1]{\textcolor{blue}{#1}}
\newcommand{\LC}[1]{\textcolor{red}{#1}}
\newcommand{\JW}[1]{\textcolor{cyan}{#1}}
  \newcounter{mnote}
\setcounter{mnote}{0}
\newcommand{\mnote}[1]{\addtocounter{mnote}{1}
	\ensuremath{{}^{\bullet\arabic{mnote}}}
	\marginpar{\footnotesize\em\color{red}\ensuremath{\bullet\arabic{mnote}}#1}}
\let\oldmarginpar\marginpar
\renewcommand\marginpar[1]{\-\oldmarginpar[\raggedleft\footnotesize #1]%
	{\raggedright\footnotesize #1}}

%\renewcommand\qedsymbol{\hfill\ensuremath{\blacksquare}}

\makeatletter\def\@captype{table}\makeatother
%\newcommand{\nm}[1]{\left\lVert {#1} \right\rVert}
%\newcommand{\proxi}[0]{ {\bf prox}}

\begin{document}
\title{Adaptive Accelerated Gradient Methods for Convex Optimization}
%\author{Long Chen}\date{\today}
\begin{abstract}
Only consider $\mu = 0$. 
\end{abstract}
\maketitle

\tableofcontents

%We focus on the case $\mu = 0$. Even for $\mu > 0$, we do not know $\mu$ or have a good guess on it. 

We discuss adaptive choice of parameters used in accelerated gradient methods. 

\section{Adaptive Accelerated Gradient Methods}
\subsection{Flow and strong Lyapunov property}
We write the first-order system
\begin{equation}\label{eq:Hagf-intro}
	\left\{
	\begin{aligned}
		x' = {}&y-x-\beta\nabla f(x),\\
		y'={}&x - y -\frac{1}{\lambda}\nabla f(x).
	\end{aligned}
	\right.
\end{equation}
%Will discuss choice of parameters 
%With adaptive choice of parameters $\beta$ and $\lambda$. 
Denote by $\bs z=(x, y)^{\intercal}$. Introduce the Lyapunov function
$$
\mathcal E(\bs z, \lambda):=f(x)-f(x^{\star})+\frac{\lambda}{2}\nm{y-x^{\star}}^2,
$$
%\end{equation}
and denote by $\mathcal G(\bs z)$ the right hand side of \eqref{eq:Hagf-intro},
%\[
%\mathcal G(\bs z, \lambda)
%%=
%%\begin{pmatrix}
%% \mathcal G^x\\
%% \mathcal G^y\\
%% \mathcal G^{\lambda}
%%\end{pmatrix}
%:=
%\begin{pmatrix}
%	y-x - \beta \nabla f(x)\\
%	\displaystyle \frac{\mu}{\lambda}(x-v)-\frac{1}{\lambda}\nabla f(x)\\
%	\mu-\lambda
%\end{pmatrix}.
%\]
which then becomes $\bs z' = \mathcal G(\bs z)$. In the notation $\nabla \mathcal E$, we consider $\lambda$ as a fixed parameter and take derivative with respect to $\bs z$. 
\begin{lemma}We have the identity
 \begin{equation}
 \begin{aligned}
	-\nabla \mathcal E(\bs z, \lambda) \cdot \mathcal G(\bs z, \lambda) = {}&
\mathcal E(\bs z, \lambda)+ \beta \nm{\nabla f(x)}_*^2  +\frac{\lambda}{2}\nm{y-x}^2 +  D_f(x^*, x) \\
&-\frac{\lambda}{2}\nm{x - x^{\star}}^2.	
\end{aligned}
\end{equation}
\end{lemma}
\begin{proof}
A direct computation gives 
\begin{equation}\label{eq:A-HNAG}
\begin{split}
	&-\nabla \mathcal E(\bs z, \lambda) \cdot \mathcal G(\bs z, \lambda) 
= 
\begin{pmatrix}
\nabla f(x)\\
 \lambda (y-x^{\star})
\end{pmatrix}
\begin{pmatrix}
(x-x^{\star}) - (y - x^{\star})+\beta\nabla f(x)\\ 
(y - x^{\star})- (x-x^{\star}) +\frac{1}{\lambda}\nabla f(x)\\ 
\end{pmatrix}
	\\
	={}& \dual{\nabla f(x),x-x^{\star}} + \beta\nm{\nabla f(x)}_*^2+ \lambda\nm{y-x^{\star}}^2 -\lambda (y-x^{\star}, x - x^{\star})\\
=	{}&  \mathcal E(\bs z, \lambda)+ \beta \nm{\nabla f(x)}_*^2  +  D_f(x^*, x)  +\frac{\lambda}{2}\nm{y-x}^2 -\frac{\lambda}{2}\nm{x - x^{\star}}^2.	
\end{split}
\end{equation}
\end{proof}

We now discuss how to cancel the negative term $- \frac{\lambda}{2} \|x-x^{\star}\|^2$ and 
then we will get the strong Lyapunov property
$$
-\nabla \mathcal E(\bs z, \lambda) \cdot \mathcal G(\bs z, \lambda) \geq \mathcal E(\bs z, \lambda).
$$
Consequently exponential stability follows.

When $f$ is $\lambda$-strongly convex, using the convexity bound  $D_f(x^*, x)\geq \frac{\lambda}{2} \|x-x^{\star}\|^2$, we can cancel the negative term and verify the strong Lyapunov property. Equivalently when $f$ is $\mu$-strongly convex, we take $\lambda =\mu$. 
%we have the last three terms
%$$
%D_f(x^*, x) + \frac{\lambda}{2}\nm{y-x^{\star}}^2-\lambda (y-x^{\star}, x - x^{\star})\geq \frac{\lambda}{2}\|x-y\|^2.
%$$

%Hence $\mathcal E$ is a strong Lyapunov function, if we can control the cross term $-\lambda (y-x^{\star}, x - x^{\star})$. 
%we can use the co-convexity 
%$$
%D_f(x^*, x)\geq \frac{1}{2L}\| \nabla f(x)\|_*^2,
%$$
%which can contribute more to the extra gradient norm component. 
%
%First by the Cauchy-Schwarz inequality,
%$$
%\lambda (y-x^{\star}, x - x^{\star})\leq \frac{\lambda}{2}\nm{y-x^{\star}}^2 + \frac{\lambda}{2}\nm{x-x^{\star}}^2. 
%$$
%To cancel the negative term $-\frac{\lambda}{2}\nm{x - x^{\star}}^2$, 
When $f$ is weakly convex only or $\mu$ is hard to estimate, 
we can choose $\lambda$ such that
\begin{equation}
\frac{\lambda}{2}\left (\nm{x-x^{\star}}^2 - \nm{x-y}^2\right )\leq \beta\| \nabla f(x)\|_*^2. 
\end{equation}
As $x^{\star}$ is unknown, we assume $\nm{x-x^{\star}}^2 \leq R_0^2$. 
% and assume we can set  $\lambda$ such that
%\begin{equation}
%\lambda \leq (2\beta + \frac{1}{L}) \frac{\| \nabla f(x)\|_*^2}{R_0^2}. 
%\end{equation}
%\LC{formulate as a theorem}

\subsection{Accelerated gradient method}\label{sec:HNAG-GS-2}
Let us apply the Gauss-Seidel type discretization to \eqref{eq:Hagf-intro} and obtain
%A Gauss-Seidel scheme without extra gradient step is considered below
\begin{equation}\label{eq:ex-HNAG}
	\left\{
	\begin{aligned}
		x_{k+1}-x_{k}={}& \alpha_k \left (y_{k}-x_{k+1}\right ) - \alpha_k \beta_k\nabla f(x_{k}),\\
 y_{k+1}-y_{k}={}&
		-\frac{\alpha_k}{\lambda_k}\nabla f(x_{k+1}) + \alpha_k \left ( x_{k+1} - y_{k+1}\right ).
	\end{aligned}
	\right.
\end{equation}
where $\alpha_k>0$ is the time step size. Denote by $L_k = (\alpha_k\beta_k)^{-1}$. 

\begin{lemma}We have the identity
 	\begin{equation*}
		%	\label{diff-Lk}
		\begin{split}
&\mathcal E(\bs z_{k+1}; \lambda_{k}) - \mathcal E(\bs z_k; \lambda_k) 
			= {} -\alpha_k\mathcal E(\bs z_{k+1}; \lambda_{k})  - \alpha_k D_f(x^{\star}, x_{k+1})\\
			&  +\frac{1}{2}\left ( \frac{\alpha_k^2}{\lambda_k}  - \frac{1}{L_k}\right )\nm{\nabla f(x_{k+1})}_*^2			\\
			& +\frac{1}{2L_k}\| \nabla f(x_{k+1}) - \nabla f(x_k) \|^2 - D_f(x_{k}, x_{k+1})\\
			& -\frac{1}{2L_k}\nm{\nabla f(x_k)}_*^2+ \frac{\alpha_k\lambda_k}{2}\left (\nm{x_{k+1}-x^{\star}}^2 - (1+\alpha_k)\nm{x_{k+1}-y_{k+1}}^2\right ).
		\end{split}
	\end{equation*}
\end{lemma}
\begin{proof}
Treat $\lambda_k$ as a fixed parameter. We expand the difference
\begin{equation}\label{eq:Elambda}
\mathcal E(\bs z_{k+1}; \lambda_{k}) - \mathcal E(\bs z_k; \lambda_k) = \langle \nabla \mathcal E(\bs z_{k+1}; \lambda_{k}), \bs z_{k+1} - \bs z_k \rangle - D_{\mathcal E}(\bs z_k, \bs z_{k+1};\lambda_k),
\end{equation}
where the negative term $- D_{\mathcal E}(\bs z_k, \bs z_{k+1};\lambda_k)$ is exapnded as
$$
- D_f(x_k, x_{k+1}) - \frac{\lambda_k}{2}\| y_{k} - y_{k+1}\|^2.
$$

Using the identity~\eqref{eq:A-HNAG} in the continuous level, we have
$$
\begin{aligned}
&\langle \nabla \mathcal E(\bs z_{k+1}; \lambda_{k}), \alpha_k \mathcal G(\bs z_{k+1}, \lambda_k) \rangle = \alpha_k \mathcal E(\bs z_{k+1}, \lambda_k)\\
&  - \frac{1}{L_k} \nm{\nabla f(x_{k+1})}_*^2  - \alpha_k D_f(x^*, x_{k+1}) + \frac{\alpha_k\lambda_k} {2}\left (\nm{x_{k+1}-x^{\star}}^2 - \nm{x_{k+1}-y_{k+1}}^2\right ).	
\end{aligned}
$$

The difference between the scheme and the implicit Euler method is
$$
\bs z_{k+1} - \bs z_k - \alpha_k \mathcal G(\bs z_{k+1}, \lambda_k) = \alpha_k 
\begin{pmatrix}
 y_k - y_{k+1} + \beta_k (\nabla f(x_{k+1}) - \nabla f(x_{k}))\\
0
\end{pmatrix}.
$$
which will bring more terms
	\begin{align*}
&		\dual{\nabla_x \mathcal E(\bs z_{k+1}, \lambda_{k}), \bs z_{k+1} - \bs z_k - \alpha_k \mathcal G(\bs z_{k+1}, \lambda_k)} \\
		&= \frac{1}{L_k} \left ( \nabla f(x_{k+1}) , \nabla f(x_{k+1}) - \nabla f(x_k)\right )+\alpha_k \dual{\nabla f(x_{k+1}), y_k - y_{k+1}}.
	\end{align*}
	We use the identity of squares for the cross term of gradients
	\begin{align*}
		&\frac{1}{L_k}( \nabla f(x_{k+1}) , \nabla f(x_{k+1}) - \nabla f(x_k)) \\
		= & - \frac{1}{2L_k}\| \nabla f(x_{k})\|_*^2
		+ \frac{1}{2L_k}\| \nabla f(x_{k+1})\|_*^2 + \frac{1}{2L_k}\| \nabla f(x_{k+1}) - \nabla f(x_k) \|_*^2.
	\end{align*}
As expected, this cross term brings more positive squares but also contribute a negative one.

On the second term, we write as
$$
\begin{aligned}
&\alpha_k \dual{\nabla f(x_{k+1}), y_k - y_{k+1}} ={} (\frac{\alpha_k}{\sqrt{\lambda_k}}\nabla f(x_{k+1}), \sqrt{\lambda_k}(y_k - y_{k+1}))\\
={}& \frac{\alpha_k^2}{2\lambda_k} \| \nabla f(x_{k+1})\|_*^2 + \frac{\lambda_k}{2}\nm{y_k - y_{k+1}}^2 - \frac{1}{2}\nm{ \frac{\alpha_k}{\sqrt{\lambda_k}}\nabla f(x_{k+1}) - \sqrt{\lambda_k} (y_{k} - y_{k+1})}^2\\
={}& \frac{\alpha_k^2}{2\lambda_k} \| \nabla f(x_{k+1})\|_*^2 + \frac{\lambda_k}{2}\nm{y_k - y_{k+1}}^2 - \frac{1}{2}\alpha_k^2\lambda_k\nm{ x_{k+1} - y_{k+1}}^2.
\end{aligned}
$$
Combining altogether, we get the desired identity. 
\end{proof}

\subsection{Adaptive choice of parameters}
Based on the identity of the difference of Lyapunov function, we propose the following adaptive procedure. 
\begin{enumerate}
\item For given $\lambda_k, L_k$, set 
\begin{equation}\label{eq:alphak}
\alpha_k = \sqrt{\frac{\lambda_k}{L_k}}. 
\end{equation}

\item Verify the inequality
\begin{equation}\label{eq:Lk}
\frac{1}{2L_k}\| \nabla f(x_{k+1}) - \nabla f(x_k) \|^2 \leq D_f(x_{k}, x_{k+1}).
\end{equation}
If not hold, increase $L_k$ by a factor. Notice that \eqref{eq:Lk} always holds with constant $L$ by the co-convexity of $f$. Once $L_k$ is updated, $\alpha_k$ should also updated by \eqref{eq:alphak}.

\item Verify the inequality
\begin{equation}\label{eq:lambdak}
 \frac{\alpha_k\lambda_k}{2}\left (R_k^2 - (1+\alpha_k)\nm{x_{k+1}-y_{k+1}}^2\right ) \leq \frac{1}{2L_k}\nm{\nabla f(x_k)}_*^2.
\end{equation}
If not hold, decrease $\lambda_k$ by a factor and repeat (1)-(3). As $L_k \leq L$ and $\alpha_k = \sqrt{\lambda_k/L_k}$, \eqref{eq:lambdak} will hold if $\lambda_k$ is sufficiently small.  

The hyper-parameter $R_k^2$ can be chosen as follows. First set a reasonable $R_0$. Then decay like $R_k = R_0/k$. It is sub-linear decay in the $\mu = 0$ case.
\end{enumerate}

\begin{theorem}
	\label{lem:conv-ex1-ode-NAG}
	For scheme \eqref{eq:ex-HNAG} with parameters satisfying \eqref{eq:alphak}, \eqref{eq:Lk}, and \eqref{eq:lambdak},  we have the linear convergence
	\begin{equation}\label{eq:conv1-ex1-ode-NAG}
\mathcal E(x_{k+1}, y_{k+1}, \lambda_{k+1}) \leq \frac{1}{1+ \sqrt{\lambda_k/L_k}} \mathcal E(x_{k}, y_{k}, \lambda_{k}). 
	\end{equation}
	where $\mathcal E_k = \mathcal E(\bs z_k, \lambda_k)
	= {}f(x_k)-f(x^{\star}) + \frac{\lambda_k}{2} \nm{y_k-x^{\star}}^2.$
\end{theorem}

\end{document}

%We start from an initial guess $(x_0, y_0)$ and set 
%$$
%\lambda_0 = \delta \|\nabla f(x_0)\|_*^{4/3}
%$$
%with a sufficiently small $\delta\in (0,1]$ s.t. $\delta \leq 1/(L^{1/3}R_0^{4/3})$.
%
%For $k=0, 1, \ldots$, given the current iterate $\bs z_k=(x_k, y_k, \lambda_k)$, one compute $x_{k+1}$ and $y_{k+1}$ successively from the first and the second equations. Notice that there might be no monotonicity on the gradient norm of iterations. We update the parameter $\lambda_{k+1}$ by
%$$
%\lambda_{k+1} = \lambda_k \times \min \left \{ \frac{\| \nabla f(x_{k+1}) \|^{4/3}}{\| \nabla f(x_{k}) \|^{4/3}}, 1 \right \}
%$$
%
%
%\LC{formulate as a lemma}
%Then it is easy to show for all $k$
%$$
%\lambda_k\leq \frac{\| \nabla f(x_k)\|^{4/3}}{L^{1/3}R_0^{4/3}}, \quad \lambda_{k+1}\leq \lambda_k.
%$$
%
%We have three parameters $(\alpha_k, \beta_k, \lambda_k)$ in \eqref{eq:ex-HNAG} and will set
%\begin{equation}\label{eq:ab}
%\alpha_k = \sqrt{\frac{\lambda_k}{L}},  \quad \alpha_k \beta_k = \frac{1}{L}.
%\end{equation}
%Note that a slight larger step size is used in \eqref{eq:ab}. 

%Although there are two gradient evaluations in the $k$-th iteration of \eqref{eq:ex-HNAG}, the second one $\nabla f(x_{k+1})$ can be reused in the $k+1$-th iteration for updating $x_{k+2}$. 

\newpage

\begin{proof}
As $\mathcal E(\bs z_{k+1}; \lambda_{k+1})\leq \mathcal E(\bs z_{k+1}; \lambda_{k})$, w
We cannot use the proof in the continuous level as in the algorithm, $x_{k+1}$ is unknown when using $\lambda_k$. 

	
In $-D_{\mathcal E}(\bs z_k, \bs z_{k+1};\lambda_k)$, we have a negative term $-\frac{\lambda_k}{2}\| y_k-y_{k+1}\|^2$ and can be used to bound the cross term $\alpha_k\dual{\nabla f(x_{k+1}), y_k - y_{k+1}}$
	\begin{equation*}
		%	\label{eq:df-diff-vk}
		\begin{split}
			\alpha_k \| \nabla f(x_{k+1})\|_*\| y_k - y_{k+1}\|\leqslant {}&
			\frac{\alpha_k^2}{2\lambda_k} \| \nabla f(x_{k+1})\|_*^2 
			+ \frac{\lambda_k}{2}\| y_k - y_{k+1}\|^2.
		\end{split}
	\end{equation*}
The benefit of HNAG is we have $- \alpha_k\beta_k \nm{\nabla f(x_{k+1})}_*^2$ from the strong Lyapunov property. 

 
	
	Adding all together, we get 
	\begin{equation*}
		%	\label{diff-Lk}
		\begin{split}
\mathcal E(\bs z_{k+1}; \lambda_{k}) - \mathcal E(\bs z_k; \lambda_k) 
			\leqslant {}& -\alpha_k\mathcal E(\bs z_{k+1}; \lambda_{k})\\
			& -\frac{\alpha_k\beta_k}{2}\nm{\nabla f(x_k)}_*^2+ \frac{\alpha_k\lambda_k}{2}\nm{x_{k+1}-x^{\star}}^2  \\
			& \quad +\frac{1}{2}\left ( \frac{\alpha_k^2}{\lambda_k}  - \alpha_k\beta_k - \frac{\alpha_k}{L}\right )\nm{\nabla f(x_{k+1})}_*^2			\\
			& \qquad+\frac{1}{2}\left ( \alpha_k\beta_k - \frac{1}{L}\right )\| \nabla f(x_{k+1}) - \nabla f(x_k) \|^2.
		\end{split}
	\end{equation*}
By our choice of parameters, the last term is vanished as $\alpha_k\beta_k = 1/L$ and the third term is negative. The step size can be slightly larger like
$$
\frac{\alpha_k^2}{\lambda_k} = \frac{1+\alpha_k}{L}.
$$
Then we chose $\lambda_k$ s.t.
$$
\frac{\alpha_k\lambda_k}{2}\nm{x_{k+1}-x^{\star}}^2 \leq \frac{1}{2L}\nm{\nabla f(x_k)}_*^2.
$$
Using the formulae of $\alpha_k$ and $\nm{x_{k+1}-x^{\star}}^2\leq R_0^2$, we can simplify the choice to
\begin{equation}\label{eq:upplambda}
\lambda_k\leq \frac{\| \nabla f(x_k)\|^{4/3}}{L^{1/3}R_0^{4/3}}.
\end{equation}

The step size 
$$
\alpha_k = \sqrt{\frac{\lambda_k}{L}} \leq \left (\frac{\| \nabla f(x_k)\|}{LR_0}\right )^{4/3}.
$$
\end{proof}	
So the convergence might be better as $\lambda_k$ is larger compare to $\|\nabla f(x_k)\|^2$. 

\subsection{Adaptive Lipschitz constant}
There are three parameters to be determined
$$
\alpha_k, L_k, \gamma_k, \quad \text{with }L_k = 1/\alpha_k\beta_k.
$$

A simple constraint is from the 
\begin{equation}
\left ( \frac{\alpha_k^2}{\lambda_k}  - \alpha_k\beta_k\right )\nm{\nabla f(x_{k+1})}_*^2,
\end{equation}
which gives relation 
$$
\alpha_k = \sqrt{\frac{\lambda_k}{L_k}} = L_k^{-4/3} d_k.
$$


To control $\nm{x_{k+1}-x^{\star}}^2$, we use
\begin{equation}
- \alpha_k D_f(x^{\star}, x_{k+1}) -\frac{1}{2L_k}\nm{\nabla f(x_k)}_*^2+ \frac{\alpha_k\lambda_k}{2}\nm{x_{k+1}-x^{\star}}^2 
\end{equation}
which put constraints
\begin{equation}\label{eq:lambdaLk}
\lambda_k \leq \frac{\| \nabla f(x_k)\|^{4/3}}{L_k^{1/3}R_0^{4/3}}.
\end{equation}
without considering the first term. We choose $\lambda_k$ by the last formulae
$$
\lambda_k = L_k^{-1/3} d_k, \quad d_k = \frac{\| \nabla f(x_k)\|^{4/3}}{R_0^{4/3}}.
$$


In the first step of updating $x_{k+1}$: 
$$
		\frac{x_{k+1}-x_{k}}{\alpha_k}= y_{k}-x_{k+1}-\beta_k\nabla f(x_{k})
$$
As $\lambda_k$ is fixed at the $k$-th iteration, 
$$
x_{k+1} = x_k+ \frac{1}{1+\alpha_k(L_k)} \left (\alpha_k(L_k) (y_k - x_k) - \frac{1}{L_k}\nabla f(x_k) \right )
$$
is a function of $L_k$. We want $L_k$ is adaptive and as small as possible to satisfy 
\begin{equation}\label{eq:Lk}
\frac{1}{2L_k}\| \nabla f(x_{k+1}) - \nabla f(x_k) \|_*^2 \leq D_f(x_k, x_{k+1}).
\end{equation}
It will hold for $L_k = L$ where $L$ is the global Lipschitz constants but can hold with much smaller $L_k$. 

Locally,  
$$
D_f(x_k, x_{k+1}) = \|x_k - x_{k+1}\|_{H(\xi)}^2, \quad \| \nabla f(x_{k+1}) - \nabla f(x_k) \|_*^2 = \|x_k - x_{k+1}\|_{H^2(\eta)}^2.
$$  
Roughly speaking, $L_k$ is the largest eigenvalue in the direction of $x_k - x_{k+1}$. If the direction corresponds to a small eigenvector, $L_k$ is small. We can use the ratio of this quantity to define a new $L_k$ and then use it to compute a new $x_{k+1}$.  

Notice that different $L_k$ values will affect not only the step size but also the direction for updating $x_{k+1}$. The search direction is a combination of $\nabla f(x_k)$ and $y_k - x_k$. Therefore, line search cannot be applied.  

The monitor or merit function for $L$ is  
$$
m(L) = L D_f(x_k, x_{k+1}(L)) - \| \nabla f(x_{k+1}(L)) - \nabla f(x_k) \|_*^2.
$$  
The local problem is  
\begin{equation}\label{eq:localproblem}
L_k = \min\{ L \mid m(L) > 0, L > 0\}.
\end{equation}  

It is better to fix $\lambda_k$ when solving the local problem \eqref{eq:localproblem}. Increasing $\lambda_k$ will increase the Lyapunov function and destroy its decreasing property. Therefore, we fix $\lambda_k$ when updating $(x_{k+1}, y_{k+1})$. However, if $L_k$ increases, we can update $\lambda_k$ because the bound becomes smaller. This process is quite tricky.  

After that, we update $\lambda_{k+1}$ using the ratio:  
\begin{equation}\label{eq:adaptive}
\lambda_{k+1} = \lambda_k \times \min \left \{ \frac{\| \nabla f(x_{k+1}) \|^{4/3}}{\| \nabla f(x_k) \|^{4/3}} \frac{L_{k-1}^{1/3}}{L_k^{1/3}}, 1 \right \}.
\end{equation}  
This formulation ensures that even if $\| \nabla f(x_{k+1}) \|$ becomes smaller, $\lambda_k$ can remain unchanged, as the local Lipschitz constant might increase.  

One more nonlinear update is applied after computing $(x_{k+1}, y_{k+1})$: we decide whether to accept $x_{k+1}$ by comparing the function values $f(x_k)$ and $f(x_{k+1})$. This rule cannot be used in the first step, as $(x, y)$ are coupled in updating $y_{k+1}$ through the second equation.  

\begin{remark}\rm  
For the strongly convex case, where $D_f(x^{\star}, x_{k+1}) \geq \frac{\mu}{2} \|x_{k+1} - x^{\star}\|^2$, we can refine the solution to satisfy  
$$
\sqrt{\lambda}(\lambda - \mu) \leq \frac{\|\nabla f(x_k)\|^2}{\sqrt{L_k}}.  
$$  
The solution is approximated by \LC{not quite correct.}  
$$
\lambda = \mu + \mu^{1/4}\|\nabla f(x_k)\|.  
$$  
If $\|\nabla f(x_k)\|^2 \geq \mu$, this approximation provides a better linear convergence rate. However, when the gradient is small, we may need to add a small adjustment to $\lambda$.  
\end{remark}  



\bibliographystyle{abbrv}
\bibliography{/Users/longchen1/Dropbox/Math/biblib/LongLibraryZotero}
\end{document}